\documentclass{beamer}

\usepackage{sams_macros_and_formatting}

%Information to be included in the title page:
\title{This is a LaTex presentation template }
\author{Sam Windels}
\institute{College of Lore}
\date{2021}

\begin{document}

\frame{\titlepage}

\begin{frame}
\frametitle{Introduction}

To enable students to easily get started with making presentations in Latex and
promote uniformity, this file provides a basic setup as well as an illustration
of the most frequently used macros.  

\end{frame}


\begin{frame}[t]
	\frametitle{Mock slide}
	\framesubtitle{Illustrating double column syntax}
	\begin{columns}
		\begin{column}{0.5\textwidth}
			This is the first column
		\end{column}
		\begin{column}{0.5\textwidth}
			This is the second column
		\end{column}
	\end{columns}
\end{frame}

\begin{frame}[t]
	\frametitle{TODOs}
	\framesubtitle{A list of elements that need to be added to this presentation}
	\begin{itemize}
		\item illustrate figures with subcaption package (https://tex.stackexchange.com/questions/122314/figures-what-is-the-difference-between-using-subfig-or-subfigure/122329)
		\item add bsc style formatting, taking inspiration from:
			\begin{itemize}
				\item https://github.com/matze/mtheme
				\item https://github.com/martinbjeldbak/ultimate-beamer-theme-list
				\item https://www.overleaf.com/learn/latex/Beamer 
			\end{itemize}

	\end{itemize}
\end{frame}

\begin{frame}[t, allowframebreaks] % `allowframebreaks splits the references over multiple slides if needed
	\frametitle{References}
	% \framesubtitle{subtitle}
	
\end{frame}

\end{document}
