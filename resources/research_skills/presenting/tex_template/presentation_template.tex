\documentclass[aspectratio=169]{beamer} % Other possible aspect ratios: 1610, 149, 54, 43 and 32. Default: 43.

\usepackage{sams_macros_and_formatting}

%% BIBLIOGRAPHY
% Add file containing references, to be rendered with biblatex. 
% To adjust formatting, see bibliography section in `sams_macros_and_formatting.sty'
\addbibresource{library.bib}

%Information to be included in the title page:
\title{This is a LaTex presentation template }
\author{Sam Windels}
\institute{Barcelona Supercomputing Centre}
\date{2021}

\begin{document}

{
\setbeamertemplate{footline}{} % To remove page number from title page
\begin{frame}
  \titlepage
\end{frame}
}
\addtocounter{framenumber}{-1} % Reset page number counting

\begin{frame}
\frametitle{Introduction}

\begin{block}{Goals}
	\begin{itemize}
		\item To enable students to easily get started with making presentations in Latex.
		\item To share best practices, from a coding perspective and slide formatting perspective.
		\item To achieve uniformity.
	\end{itemize} 
\end{block}

\end{frame}

\begin{frame}[t]
	\frametitle{Illustrating double column syntax}
	\begin{columns}
		\begin{column}{0.5\textwidth}
			This is the first column
		\end{column}
		\begin{column}{0.5\textwidth}
			This is the second column
		\end{column}
	\end{columns}
\end{frame}

\begin{frame}[t]
	\frametitle{Illustrating blocks}

	\begin{block}{This is a `block'}
		This is text in a block.
	\end{block}

	\begin{exampleblock}{This is an `exampleblock'}
		\begin{itemize}
			\item This is an item list, nested in a example block. Notice the colour.
			\item[*] Note that you can change the icon used to indicate an item, to a star for instance.
		\end{itemize}		
	\end{exampleblock}
	
	\begin{alertblock}{This is an `alertblock'}
		\begin{enumerate}
			\item Mind you, if you want numbered a numbered list, you better use enumerate.
			\item You see?
		\end{enumerate}
	\end{alertblock}
	
\end{frame}

\begin{frame}[t]
	\frametitle{TODOs}
	\framesubtitle{A list of elements that need to be added to this presentation}
	\begin{itemize}
		\item illustrate figures with subcaption package (https://tex.stackexchange.com/questions/122314/figures-what-is-the-difference-between-using-subfig-or-subfigure/122329)
		\item add bsc style theme, taking inspiration from:
		\begin{itemize}
			\item https://github.com/matze/mtheme
			\item https://github.com/martinbjeldbak/ultimate-beamer-theme-list
			\item https://www.overleaf.com/learn/latex/Beamer 
		\end{itemize}
		will require us to format `blocks'
	\item illustrate sections, subsections and outline frame 
	\end{itemize}
\end{frame}

\begin{frame}[t]
	\frametitle{Table illustration}

	\begin{table}[htpb]
		\centering
		\begin{tabular}{cc}
			Towards a data integrated cell & \cite{malod2019towards}	\\
			Learning the parts of objects by non-negative matrix factorization & \cite{Lee1999}	
		\end{tabular}
		\caption{{\bf Some key NMTF papers.}}
		\label{tab:table}
	\end{table}	
\end{frame}

\begin{frame}[t, allowframebreaks] % `allowframebreaks' automatically splits the reference list over multiple slides if needed
	\frametitle{References}
	
	\printbibliography

\end{frame}

\end{document}
