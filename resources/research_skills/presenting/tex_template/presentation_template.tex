\documentclass[aspectratio=169]{beamer} % Other possible aspect ratios: 1610, 149, 54, 43 and 32. Default: 43.

\usepackage{sams_macros_and_formatting}

%% BIBLIOGRAPHY
% Add file containing references, to be rendered with biblatex. 
% To adjust formatting, see bibliography section in `sams_macros_and_formatting.sty'
\addbibresource{library.bib}

%Information to be included in the title page:
\title{This is a LaTex presentation template }
\subtitle{Providing an accessible intro to LaTex presentations}

\author{Sam Windels}
\institute{Barcelona Supercomputing Centre}
\date{\today}

\begin{document}

{
\setbeamertemplate{footline}{} % To remove page number from title page
\begin{frame}
  \titlepage
\end{frame}
}
\addtocounter{framenumber}{-1} % Reset page number counting

\begin{frame}[c]
	\frametitle{Outline}
 	\tableofcontents
\end{frame}


\section{Introduction}	

\begin{frame}[c]
	\sectionpage
\end{frame}

\begin{frame}
\frametitle{Motivation}

\begin{block}{Goals}
	\begin{itemize}
		\item To enable students to easily get started with making presentations in Latex.
		\item To share coding best-practices.
	\end{itemize} 
\end{block}

\end{frame}


\section{Examples}	

\begin{frame}[c]
	\sectionpage
\end{frame}

\begin{frame}[c]
	\frametitle{Layout example}
	This slide is vertically centred. 	
\end{frame}


\begin{frame}[c]
	\frametitle{Layout example}
	\centering This slide is completely centred. 
\end{frame}

\begin{frame}[t]
	\frametitle{Layout example}
	In this slide, everything is moved to the top. 	
\end{frame}

\begin{frame}[c]
	\frametitle{Illustrating blocks}

	\begin{block}{This is a `block'}
		This is text in a block.
	\end{block}

	\begin{exampleblock}{This is an `exampleblock'}
		This is a different type of block. Notice the colour?
		% \begin{itemize}
		% 	\item This is an item list, nested in a example block. 
		% 	\item[*] Note that you can change the icon used to indicate an item, to a star for instance.
		% \end{itemize}		
	\end{exampleblock}
	
	\begin{alertblock}{This is an `alertblock'}
		This is yet another different type of block. 
		% \begin{enumerate}
		% 	\item If you want numbered a numbered list, you better use enumerate.
		% 	\item You see?
		% \end{enumerate}
	\end{alertblock}
	
\end{frame}

\begin{frame}[t]
	\frametitle{Illustrating of double columns}
	\begin{columns}
		\begin{column}{0.5\textwidth}
			\begin{figure}[htpb]
				\centering
				\includegraphics[width=0.8\linewidth]{duck_a.jpeg}
				\caption{{\bf Some random duck}}%
			\end{figure}
		\end{column}
		\begin{column}{0.5\textwidth}
			{\bf Wikipedia entry about ducks} \\
			`Duck' is the common name for numerous species of
			waterfowl in the family Anatidae. Ducks are generally
			smaller and shorter-necked than swans and geese, which
			are members of the same family. Divided among several
			subfamilies, they are a form taxon; they do not
			represent a monophyletic group (the group of all
			descendants of a single common ancestral species),
			since swans and geese are not considered ducks. Ducks
			are mostly aquatic birds, and may be found in both
			fresh water and sea water.
		\end{column} 
	\end{columns} 
\end{frame}


\begin{frame}[t]
	\frametitle{Illustration of sub-figures}
	\begin{figure}[htpb]
		\centering
		\begin{tabular}{cc}
			(A) \includegraphics[width=0.28\linewidth]{duck_a.jpeg} &
			(B) \includegraphics[width=0.28\linewidth]{duck_b.jpeg} \\
		\end{tabular}
		\caption{{\bf Plotting two random ducks}}%
	\end{figure}
	An easy way to plot two sub-figures is to simply nest them in a table.\\
	This will do for now.
	
\end{frame}

\begin{frame}[c]
	\frametitle{Table illustration}

	\begin{table}[htpb]
		\centering
		\begin{tabular}{|c|c|}
			\hline
			{\bf Title} 								& {\bf reference} 	 \\
			\hline
			Towards a data integrated cell 						& \cite{malod2019towards}\\
			Learning the parts of objects by non-negative matrix factorization 	& \cite{Lee1999}	 \\
			\hline
		\end{tabular}
		\caption{{\bf Some key NMTF papers.}}
		% \label{tab:table}
	\end{table}	
	Websites to make LaTex tables in a GUI:
	\begin{itemize}
		\item tablesgenerator.com 
		\item latex-tables.com
	\end{itemize}
\end{frame}

\begin{frame}[t, allowframebreaks] % `allowframebreaks' automatically splits the reference list over multiple slides if needed
	\frametitle{References}
	
	\printbibliography

\end{frame}

\end{document}
